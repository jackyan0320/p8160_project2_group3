\PassOptionsToPackage{unicode=true}{hyperref} % options for packages loaded elsewhere
\PassOptionsToPackage{hyphens}{url}
%
\documentclass[]{article}
\usepackage{lmodern}
\usepackage{amssymb,amsmath}
\usepackage{ifxetex,ifluatex}
\usepackage{fixltx2e} % provides \textsubscript
\ifnum 0\ifxetex 1\fi\ifluatex 1\fi=0 % if pdftex
  \usepackage[T1]{fontenc}
  \usepackage[utf8]{inputenc}
  \usepackage{textcomp} % provides euro and other symbols
\else % if luatex or xelatex
  \usepackage{unicode-math}
  \defaultfontfeatures{Ligatures=TeX,Scale=MatchLowercase}
\fi
% use upquote if available, for straight quotes in verbatim environments
\IfFileExists{upquote.sty}{\usepackage{upquote}}{}
% use microtype if available
\IfFileExists{microtype.sty}{%
\usepackage[]{microtype}
\UseMicrotypeSet[protrusion]{basicmath} % disable protrusion for tt fonts
}{}
\IfFileExists{parskip.sty}{%
\usepackage{parskip}
}{% else
\setlength{\parindent}{0pt}
\setlength{\parskip}{6pt plus 2pt minus 1pt}
}
\usepackage{hyperref}
\hypersetup{
            pdftitle={Breast Cancer Final Report},
            pdfborder={0 0 0},
            breaklinks=true}
\urlstyle{same}  % don't use monospace font for urls
\usepackage[margin=1in]{geometry}
\usepackage{graphicx,grffile}
\makeatletter
\def\maxwidth{\ifdim\Gin@nat@width>\linewidth\linewidth\else\Gin@nat@width\fi}
\def\maxheight{\ifdim\Gin@nat@height>\textheight\textheight\else\Gin@nat@height\fi}
\makeatother
% Scale images if necessary, so that they will not overflow the page
% margins by default, and it is still possible to overwrite the defaults
% using explicit options in \includegraphics[width, height, ...]{}
\setkeys{Gin}{width=\maxwidth,height=\maxheight,keepaspectratio}
\setlength{\emergencystretch}{3em}  % prevent overfull lines
\providecommand{\tightlist}{%
  \setlength{\itemsep}{0pt}\setlength{\parskip}{0pt}}
\setcounter{secnumdepth}{0}
% Redefines (sub)paragraphs to behave more like sections
\ifx\paragraph\undefined\else
\let\oldparagraph\paragraph
\renewcommand{\paragraph}[1]{\oldparagraph{#1}\mbox{}}
\fi
\ifx\subparagraph\undefined\else
\let\oldsubparagraph\subparagraph
\renewcommand{\subparagraph}[1]{\oldsubparagraph{#1}\mbox{}}
\fi

% set default figure placement to htbp
\makeatletter
\def\fps@figure{htbp}
\makeatother


\title{Breast Cancer Final Report}
\author{}
\date{\vspace{-2.5em}}

\begin{document}
\maketitle

\#Introduction

In this project, we build a model to classify breast tissue images into
malignant cancer or benign tissue. We are interested in observing how
the certain features for individual breast tissue are influential in
determining cancer diagnosis. First, we built a full logistic model with
all 30 features and utilized a Newton-Raphson algorithm in order to
estimate the model coefficients. Furthermore, we built a logistic-lasso
model to perform variable selection in order to identify the best subset
of tissue characteristics that play a major role in prediction.

\hypertarget{method}{%
\section{Method}\label{method}}

The dataset contains 569 individuals with 30 variables to help predict
malignancy. The 30 variables correspond to mean, standard deviation, and
largest value of 10 tissue characteristics, including, radius, texture,
perimeter, area, smoothness, compactness, concavity, concave points,
symmetry, and fractal dimension.

\hypertarget{logistic-regression-model}{%
\subsection{Logistic Regression Model}\label{logistic-regression-model}}

The logistics regression model is

\[
\log(\frac{\pi}{1-\pi})=X\beta
\]

where the link function is \(\log(\frac{\pi}{1-\pi})\).

The likelihood function is

\[
L(\beta ; X, y)=\prod_{i=1}^{n}\left\{\left(\frac{\exp \left(X_{i} \beta\right)}{1+\exp \left(X_{i} \beta\right)}\right)^{y_{i}}\left(\frac{1}{1+\exp \left(X_{i} \beta\right)}\right)^{1-y_{i}}\right\}
\] where \(y_i\) is the indicator of responses ``M'' or ``B''.

Consequently, the log-likelihood function is

\[
l(\beta)=\sum_{i=1}^{n}\left\{y_{i}\left(X_{i} \beta\right)-\log \left(1+\exp \left(X_{i} \beta\right)\right)\right\}
\] Maximizing log-likelihood function is equivalent to maximizing
likelihood function, so the gradient and Hessian matrix will be

\[
\nabla l\left(\boldsymbol{\beta}\right)=\left(\begin{array}{c}\sum_{i=1}^{n} y_{i}-p_{i} \\ \sum_{i=1}^{n} \mathbf{x}_{i}\left(y_{i}-p_{i}\right)\end{array}\right)_{(p+1) \times 1}
\] , and

\[
\begin{aligned} \nabla^{2} l\left(\boldsymbol{\beta}\right) &=-\sum_{i=1}^{n}\left(\begin{array}{c}1 \\ \mathbf{x}_{i}\end{array}\right)\left(\begin{array}{cc}1 & \mathbf{x}_{i}^{T}\end{array}\right) p_{i}\left(1-p_{i}\right) \\ &=-\left(\begin{array}{cc}\sum p_{i}\left(1-p_{i}\right) & \sum \mathbf{x}_{i}^{T} p_{i}\left(1-p_{i}\right) \\ \sum \mathbf{x}_{i} p_{i}\left(1-p_{i}\right) & \sum \mathbf{x}_{i} \mathbf{x}_{i}^{T} p_{i}\left(1-p_{i}\right)\end{array}\right) \end{aligned}
\] where
\(p_{i}=P\left(Y_{i}=1 | \mathbf{x}_{i}\right)=\frac{\exp \left(\beta_{0}+\mathbf{x}_{i}^{T} \boldsymbol{\beta}_{1}\right)}{1+\exp \left(\beta_{0}+\mathbf{x}_{i}^{T} \boldsymbol{\beta}_{1}\right)}\).

\hypertarget{algorithms}{%
\subsection{Algorithms}\label{algorithms}}

\hypertarget{newton-raphson-algorithm}{%
\subsubsection{Newton-Raphson
Algorithm}\label{newton-raphson-algorithm}}

Newton-Raphson algorithm is a method to search for solutions to the
system of equations \(\nabla l\left(\boldsymbol{\beta}\right)=0\). At
each step, given the current point \(\boldsymbol{\beta}_0\), the
gradient \(\nabla l\left(\boldsymbol{\beta}\right)\) for
\(\boldsymbol{\beta}\) near \(\boldsymbol{\beta}_0\) may be approximated
by

\[
\nabla l\left(\boldsymbol{\beta}_{0}\right)+\nabla^{2} l\left(\boldsymbol{\beta}_{0}\right)\left(\boldsymbol{\beta}-\boldsymbol{\beta}_{0}\right)
\]

The next step in the algorithm is determined by solving the system of
linear equations

\[
\nabla l\left(\boldsymbol{\beta}_{0}\right)+\nabla^{2} l\left(\boldsymbol{\beta}_{0}\right)\left(\boldsymbol{\beta}-\boldsymbol{\beta}_{0}\right)=\mathbf{0}
\] and the next ``current point'' is set to be the solution, which is

\[
\boldsymbol{\beta}_{1}=\boldsymbol{\beta}_{0}-\left[\nabla^{2} l\left(\boldsymbol{\beta}_{0}\right)\right]^{-1} \nabla l\left(\boldsymbol{\beta}_{0}\right)
\]

The ith step is given by

\[
\boldsymbol{\beta}_{i}=\boldsymbol{\beta}_{i-1}-\left[\nabla^{2} l\left(\boldsymbol{\beta}_{i-1}\right)\right]^{-1} \nabla l\left(\boldsymbol{\beta}_{i-1}\right)
\]

\hypertarget{pathwise-coordinate-descent-with-regularized-logistic-regression}{%
\subsubsection{Pathwise Coordinate Descent with regularized logistic
regression}\label{pathwise-coordinate-descent-with-regularized-logistic-regression}}

The logistic-lasso can be written as a penalized weighted least-squares
problem

\[
\min _{\left(\beta_{0}, \boldsymbol{\beta}_{1}\right)} L\left(\beta_{0}, \boldsymbol{\beta}_{1}, \lambda\right)=\left\{-\ell\left(\beta_{0}, \boldsymbol{\beta}_{1}\right)+\lambda \sum_{j=0}^{p}\left|\beta_{j}\right|\right\}
\]

When the p, the number of parameters, is large, the optimization could
be challenging. Therefore, a coordinate-wise descent algorithm will be
applied. The objective function is

\[
f\left(\beta_{j}\right)=\frac{1}{2} \sum_{i=1}^{n}\left(y_{i}-\sum_{k \neq j} x_{i, k} \widetilde{\beta}_{k}-x_{i, j} \beta_{j}\right)^{2}+\gamma \sum_{k \neq j}\left|\widetilde{\beta}_{k}\right|+\gamma\left|\beta_{j}\right|
\]

Minimizing \(f\left(\beta_{j}\right)\) w.r.t \(\beta_{j}\) while having
\(\widetilde{\beta}_{k}\) fixed, we have weighted beta updates

\[
\widetilde{\beta}_{j}(\gamma) \leftarrow \frac{S\left(\sum_{i} w_{i} x_{i, j}\left(y_{i}-\tilde{y}_{i}^{(-j)}\right), \gamma\right)}{\sum_{i} w_{i} x_{i, j}^{2}}
\]

where
\(\tilde{y}_{i}^{(-j)}=\sum_{k \neq j} x_{i, k} \widetilde{\beta}_{k}\).

If we Taylor expansion the log-likelihood around ``current estimates''
\(\left(\widetilde{\beta}_{0}, \tilde{\beta}_{1}\right)\), we have a
quadratic approximation to the log-likelihood

\[
f\left(\beta_{0}, \boldsymbol{\beta}_{1}\right) \approx \ell\left(\beta_{0}, \boldsymbol{\beta}_{1}\right)=-\frac{1}{2 n} \sum_{i=1}^{n} w_{i}\left(z_{i}-\beta_{0}-\mathbf{x}_{i}^{T} \boldsymbol{\beta}_{1}\right)^{2}+C\left(\widetilde{\beta}_{0}, \widetilde{\boldsymbol{\beta}}_{1}\right)
\] where \[
z_{i}=\widetilde{\beta}_{0}+\mathbf{x}_{i}^{T} \widetilde{\boldsymbol{\beta}}_{1}+\frac{y_{i}-\widetilde{p}_{i}\left(\mathbf{x}_{i}\right)}{\widetilde{p}_{i}\left(\mathbf{x}_{i}\right)\left(1-\widetilde{p}_{i}\left(\mathbf{x}_{i}\right)\right)}
\]

\[
w_{i}=\widetilde{p}_{i}\left(\mathbf{x}_{i}\right)\left(1-\widetilde{p}_{i}\left(\mathbf{x}_{i}\right)\right)
\]

\[
\widetilde{p}_{i}=\frac{\exp \left(\widetilde{\beta}_{0}+\mathbf{x}_{i}^{T} \widetilde{\boldsymbol{\beta}}\right)}{1+\exp \left(\widetilde{\beta}_{0}+\mathbf{x}_{i}^{T} \widetilde{\boldsymbol{\beta}}_{1}\right)}
\] Therefore, the algorithms steps are

Step 1. Find \(\lambda_{max}\) such that all the estimatedβare zero;

Step 2. Define a fine sequence
\(\lambda_{max}\ge\lambda_1\ge...\ge\lambda_{min}\ge0\);

Step 3. Defined the quadratic approximated objective function
\(L\left(\beta_{0}, \boldsymbol{\beta}_{1}, \lambda\right)\) for
\(\lambda_k\) using the estimated parameter at
\(\lambda_{k-1}\;\left(\lambda_{k-1}>\lambda_{k}\right)\).

Step 4. Run coordinate descendent algorithm to find the
optimizationdefined in Step 3

\end{document}
